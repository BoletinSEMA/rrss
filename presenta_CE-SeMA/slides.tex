
\begin{frame}
  \titlepage
\end{frame}

\begin{frame}
  \frametitle{Redes Sociales}
  \begin{itemize}
    \item Nadie duda de la importancia de las RRSS hoy en día
    \item ... especialmente, entre los más ¡\textit{jóvenes}!
    \item Facebook, Instagram...
    \pause
    y \textbf{Twitter}
  \end{itemize}
\end{frame}

\begin{frame}
  \frametitle{Organizaciones Matemáticas Españolas en Twitter}
  \begin{itemize}
  \item Institutos de investigación:\par
    \begin{small}
    \textbf{ICMAT}: \texttt{\makeatletter @\_ICMAT} (desde 2012, \textit{24.800} seguidores),
      \\
    \textbf{IMUS}: \texttt{\makeatletter @imus\_us} (desde 2017, \textit{653} seguidores), ...
  \end{small}
  \item Sociedades Matemáticas.\par
    \begin{small}
      \textbf{RSME}: \texttt{\makeatletter @RealSocMatEsp} (desde 2013, \textit{7.391} seguidores),
      \\
    \textbf{SEIO}: \texttt{\makeatletter @seio\_es} (desde 2012, \textit{804} seguidores),...
      \\
      \pause
      % \href{https://twitter.com/sema_informa}{\textbf{SeMA_XYZ}}: \texttt{\makeatletter @sema\_} (desde 2019, \textit{28} seguidores en 2 días!),...
      \href{https://twitter.com/sema\_informa}{\textbf{SeMA}}:
      \texttt{\makeatletter @sema\_XYZ} (desde 2019, \textit{31}
      seguidores en 3 días!),...
  \end{small}
  \end{itemize}
\end{frame}

\begin{frame}
  \singlesubheading{
  ¿Quién le pone el cascabel al gato?
  }
\end{frame}

\begin{frame}{Dificultades}
  \begin{itemize}
  \item ``Analfabetismo'' de los menos jóvenes en RRSS
  \item Puesta a punto de la infraestructura de RRSS
  \item Dedicación constante en el tiempo (lo más difícil!)
    \begin{center}
      \textit{Publicación de información \textbf{diraria}}
    \end{center}
  \end{itemize}
\end{frame}

\begin{frame}
  \singlesubheading{
    Idea: el equipo de edición del \textit{Boletín Electrónico}
  }
\end{frame}

\begin{frame}{Boletín Electrónico de la SeMA}
  \begin{flushright}
  \href{https://www.sema.org.es/images/boletines/boletin23/23}{[... enlace Boletín 23]}
\end{flushright}
  \begin{itemize}
  \item Paco Ortegón: coordinación, versión PDF
  \item \structure<2->{Rafa Rguez. Galván: versión web}
  \item Colaboradores:
    \begin{itemize}
    \item \structure<2-3>{Alba Navarro},
      \structure<2-3>{Noelia Ortega},
      \structure<2-3>{Daniel Acosta},
      \structure<2-3>{Juan Antonio Guitarte}
      \par
      \begin{flushright}\small
        (socios de SeMA)
      \end{flushright}
    \item \textit{\structure<2->{Gloria Almozara}}
      \par
      \begin{flushright}\small
      (no socia)
    \end{flushright}
    \end{itemize}
  \item<3>$+$ \structure<3>{Paco Arándiga}
  \end{itemize}
\end{frame}

\begin{frame}
  \singlesubheading{
  Y ¿cómo lo hacemos?
  }
\end{frame}


\begin{frame}{Etapa 1. Creación de Infraestructuras}
    \begin{itemize}
    \item  Twitter $+$ Instagram $+$ Facebook
    \item Elección de un \textbf{nombre de usuario común} (difícil!)
    \item Algunas posibilidades:
      \begin{itemize}
      \item \texttt{SeMA\_informa}
      \item \texttt{SeMA\_mat\_apl}
      \item \texttt{SeMA\_divulga}
      \end{itemize}
    \item Estado actual (\textit{twiter}): \par
      \begin{flushright}
      \url{https://twitter.com/sema\_informa}
    \end{flushright}
  \item Orientación hacia jóvenes: ``conjugar perfil serio y perfil informal''
    \end{itemize}
\end{frame}

\begin{frame}{Etapa 2. Mantenimiento en el Tiempo}
  \textbf{Publicación diaria de contenidos}
    \begin{itemize}
    \item  \structure{Contenidos generados por terceros}
      \par... ¿quienes? ¿cómo? (política de RRSS)
    \item \structure{Contenidos propios}
      \begin{itemize}
      \item Coordinación con el Boletín (noticias,
        avisos...  \textit{mayor agilidad})
      \item {\textbf{Delegados de la SeMA} en universidades} (Boletín y RRSS!!)
        \begin{itemize}
        \item Enviar noticias locales relevantes de su propia
          universidad
        \item Difundir las que se produzcan desde la SEMA hacia su universidad
        \end{itemize}
      \end{itemize}
    \item ...
    \end{itemize}
\end{frame}

\begin{frame}
  \singlesubheading{
    ...mucho por hacer.
  }
\end{frame}

% \begin{frame}
%   \singletitle{Section One}
%   \notelist{
%     \item Here begins section one
%   }
% \end{frame}

% \begin{frame}
%   \singlesubheading{
%     I want to emphasize this
%   }
%   \notelist{
%     \item This is super important
%   }
% \end{frame}

% \begin{frame}
%   \singletext{This is well, but no so much.}
%   \notelist{
%     \item Let's say it anyway.
%   }
% \end{frame}
